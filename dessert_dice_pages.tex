\documentclass[a6paper, 11pt, parskip=half, DIV=15]{scrartcl}
\usepackage{unicode-math}
\setmathfont{TexGyreSchola-Math}
\usepackage{dessertdice}
\usepackage{caption}
\usepackage{ragged2e}
% Minimize unwanted hyphenation
\tolerance=1
\emergencystretch=\maxdimen
\hyphenpenalty=1
\hbadness=10000

\usepackage{eso-pic}
\usepackage{booktabs}

\setkomafont{section}{\setmainfont{LondrinaSolid}\LARGE\color{IceCreamPurple}}
\setkomafont{subsection}{\setmainfont{LondrinaSolid}\Large\color{IceCreamPurple}}
\setkomafont{subsubsection}{\setmainfont{LondrinaSolid}\large\color{IceCreamPurple}}

% Adjust spacing before and after section headings
\RedeclareSectionCommand[
  runin=false,
  beforeskip=0.5\baselineskip,
  afterskip=-0.0\baselineskip
]{section}

% Adjust spacing before and after subsection headings
\RedeclareSectionCommand[
  runin=false,
  beforeskip=0.5\baselineskip,
  afterskip=-0.0\baselineskip
]{subsection}

% Adjust spacing before and after subsubsection headings
\RedeclareSectionCommand[
  runin=false,
  beforeskip=0.5\baselineskip,
  afterskip=-0.0\baselineskip
]{subsubsection}


\usepackage{enumitem}

\usepackage[hang,flushmargin]{footmisc}
\newcommand\blfootnote[1]{%
  \begingroup
  \renewcommand\thefootnote{}\footnote{#1}%
  \addtocounter{footnote}{-1}%
  \endgroup
}

\renewcommand{\thefootnote}{\fnsymbol{footnote}}
\renewcommand{\footnoterule}{%
  \kern -3pt
  \hrule width \textwidth height 0.5pt
  \kern 2pt
}

\usepackage[hidelinks]{hyperref}
\usepackage[type={CC}, version={4.0}, modifier={by-sa}]{doclicense} % Add text and icons for creative commons license
%\usepackage{array}

\raggedright
\pagestyle{empty}
\begin{document}

\begin{titlepage}
\AddToShipoutPictureBG{
\begin{tikzpicture}[remember picture, overlay]
%	\node () at (current page.center) {\includegraphics[width=\pagewidth, height=\pageheight]{Images/aloft_cover_background.png}};
	\node () at (current page.center) {\includegraphics[width=\pagewidth, height=\pageheight]{Images/dessert_dice_front_cover_compressed.jpg}};
\end{tikzpicture}
}

\enlargethispage{3.5\baselineskip}
\setmainfont[Scale=2.75]{LondrinaSolid}
%\setmainfont[Scale=1.9]{LondrinaSolid-Black}
\Huge
\phantom{a}
\end{titlepage}


\ClearShipoutPicture
\enlargethispage{1.75\baselineskip}
\section*{Overview}
Dessert Dice is a delicious dice-placement game for two to four players. It can be played in about fifteen minutes and is intended for players who are at least eight years old.

\section*{Components}
\begin{itemize}[leftmargin=*]
  \item \textbf{25 dice}: 6 red dice, 6 yellow dice, 6 green dice, 6 purple dice, and 1 black die.
%    \begin{itemize}[nosep, leftmargin=*]
%      \item 6 red dice, 6 yellow dice, 6 green dice, 6~purple dice, and 1 black die.
%    \end{itemize}
  \item \textbf{4 cards}: 1 Jello Tart card, 1 Sweet Roll card, 1 Popsicle card, and 1 Ice Cream card.
%      \begin{itemize}[nosep, leftmargin=*]
%      \item 1 Jello Tart card, 1 Sweet Roll card, 1 Popsicle card, and 1 Ice Cream card.
%    \end{itemize}
  \item \textbf{1 board}: A 5$\times$5 grid of squares.
\end{itemize}

\section*{Set Up}
\begin{enumerate}[leftmargin=*]
  \item Place the board in the play area.
  \item Deal one card to each player. Look at your card but don't show it to anyone else.
  \item Roll the dice. These dice form the \emph{supply}.
  \item Decide who will take the first turn. The person who most recently finished all of their vegetables should go first. 
 \end{enumerate}
 
\newpage
\enlargethispage{1.75\baselineskip}
\section*{Gameplay}
During the game, you will take turns. On your turn you will either place a die from the supply onto the board or move a die from one square on the board to another.

After you take your turn, the player to your left should go next. That is, you should take turns in clockwise order.

The game ends when the board is full. 

\subsection*{Place a Die}
To place a die, take any die from the supply and place it in any empty square on the board.

\begin{itemize}
	\item You may not change which face of the die is pointed upwards when you place a die.
\end{itemize}

\newpage
\enlargethispage{1.75\baselineskip}
\subsection*{Move a Die}
Two squares are \emph{neighbors} if they share an edge. Squares that share only a corner are not neighbors.

To move a die, tip it into an empty neighboring square on the board. Note that moving a die changes both the square the die is in and the face that is pointed upwards.

\begin{itemize}
    \item You may not move a die off of the board.
    \item You may not move a die diagonally.
    \item If none of a die's neighboring squares are empty, then you may not move that die.
    \item If the previous player moved a die on their turn, then you may not undo their action by moving that same die back to the square it was in at the start of their turn.
\end{itemize}

\newpage
\enlargethispage{1.75\baselineskip}
\section*{Scoring}
The dessert icon on your card determines which of the dice will be used to compute your score at the end of the game.

The following terminology is used to describe how to compute your score:
\begin{itemize}
	\item An icon is \emph{displayed} on a die if it appears on the face that is pointed upwards.
    \item Two dice are \emph{adjacent} if they are in two neighboring squares on the board.
    \item A \emph{cluster} is a set of dice on which the same icon is displayed and each die in the set is adjacent to another die in the set. 
\end{itemize}

Your score is the size of the largest cluster of dice on which your dessert icon is displayed.

You win if you have the highest score. In the case of a tie, the highest-scoring player who most recently placed a die on their turn wins.

\vfill
\hrulefill

\textbf{Contact}: \href{mailto:dessert.dice.game@gmail.com}{dessert.dice.game@gmail.com}\\
\textbf{License}: This work is licensed under a\\\phantom{\textbf{License}: }``CC BY-SA 4.0'' license.%\raggedright\doclicenseText
 \newpage
 \AddToShipoutPictureBG{
\begin{tikzpicture}[remember picture, overlay]
%	\node () at (current page.center) {\includegraphics[width=\pagewidth, height=\pageheight]{Images/aloft_cover_background.png}};
	\node () at (current page.center) {\includegraphics[width=\pagewidth, height=\pageheight]{Images/dessert_dice_back_cover_compressed.jpg}};
\end{tikzpicture}
}
\phantom{Dessert Dice}
\end{document}